
\documentclass[12pt]{article}
\usepackage[notext]{stix}
\usepackage{step}
\usepackage[T1]{fontenc}
\usepackage{amsmath}
\usepackage{amsfonts}
\usepackage[margin=2.5cm]{geometry}
\usepackage{titling}
\usepackage{listings}
\usepackage{titlesec}
\usepackage{parskip}


\renewcommand\maketitlehooka{\null\mbox{}\vfill}
\renewcommand\maketitlehookd{\vfill\null}

\title{Operating Systems Notes}

\author{SubmergedDuck}
\date{\today}

\begin{document}
\begin{titlingpage}
\maketitle 
\end{titlingpage}

\newpage

\section{Page Tables {\hrule}}
\vspace{-1.5em}

{\bf Pages.} {Memory is split into pages of equal size. Inside one page,
the offset tells "which byte within this page."} \par 

\noindent
{\bf Offset Bits.} {How many bits to count all bytes in one page.} \par  

\setlength{\parindent}{15pt}  {\bf Eg.} {A 4KB page has $4096$ bytes. Since $4096 = 2^{12}$, the page offset uses 12 bits.} \par \vspace{1em}

\noindent{\bf Virtual Page Number (VPN).} {Given a 32-bit virtual address and 4KB pages, there would be 20 bits left for the VPN. This implies that there are $2^{20}$ virtual pages if there are 20 VPN bits.}

\setlength{\parindent}{0pt} 
{\bf Page Table Entry (PTE).} {One record in the page table that maps a virtual page to a physical frame and bits like valid/present, R/W/X permissions, dirty, accessed, etc.}









\vspace{10em}
\begin{lstlisting}[language=C, numbers=left, basicstyle=\ttfamily]
def hello_world():
    print("Hello, LaTeX!")
\end{lstlisting}



\end{document}